% THIS IS SIGPROC-SP.TEX - VERSION 3.1
% WORKS WITH V3.2SP OF ACM_PROC_ARTICLE-SP.CLS
% APRIL 2009
%
% It is an example file showing how to use the 'acm_proc_article-sp.cls' V3.2SP
% LaTeX2e document class file for Conference Proceedings submissions.
% ----------------------------------------------------------------------------------------------------------------
% This .tex file (and associated .cls V3.2SP) *DOES NOT* produce:
%       1) The Permission Statement
%       2) The Conference (location) Info information
%       3) The Copyright Line with ACM data
%       4) Page numbering
% ---------------------------------------------------------------------------------------------------------------
% It is an example which *does* use the .bib file (from which the .bbl file
% is produced).
% REMEMBER HOWEVER: After having produced the .bbl file,
% and prior to final submission,
% you need to 'insert'  your .bbl file into your source .tex file so as to provide
% ONE 'self-contained' source file.
%
% Questions regarding SIGS should be sent to
% Adrienne Griscti ---> griscti@acm.org
%
% Questions/suggestions regarding the guidelines, .tex and .cls files, etc. to
% Gerald Murray ---> murray@hq.acm.org
%
% For tracking purposes - this is V3.1SP - APRIL 2009

\documentclass{acm_proc_article-sp}
\usepackage{amsmath}
\newcommand{\argmax}{\mathop{\rm arg~max}\limits}
\newcommand{\argmin}{\mathop{\rm arg~min}\limits}
\begin{document}

\title{Value of a Click}
\subtitle{[Extended Abstract]}
%
% You need the command \numberofauthors to handle the 'placement
% and alignment' of the authors beneath the title.
%
% For aesthetic reasons, we recommend 'three authors at a time'
% i.e. three 'name/affiliation blocks' be placed beneath the title.
%
% NOTE: You are NOT restricted in how many 'rows' of
% "name/affiliations" may appear. We just ask that you restrict
% the number of 'columns' to three.
%
% Because of the available 'opening page real-estate'
% we ask you to refrain from putting more than six authors
% (two rows with three columns) beneath the article title.
% More than six makes the first-page appear very cluttered indeed.
%
% Use the \alignauthor commands to handle the names
% and affiliations for an 'aesthetic maximum' of six authors.
% Add names, affiliations, addresses for
% the seventh etc. author(s) as the argument for the
% \additionalauthors command.
% These 'additional authors' will be output/set for you
% without further effort on your part as the last section in
% the body of your article BEFORE References or any Appendices.

\numberofauthors{8} %  in this sample file, there are a *total*
% of EIGHT authors. SIX appear on the 'first-page' (for formatting
% reasons) and the remaining two appear in the \additionalauthors section.
%
\author{
% You can go ahead and credit any number of authors here,
% e.g. one 'row of three' or two rows (consisting of one row of three
% and a second row of one, two or three).
%
% The command \alignauthor (no curly braces needed) should
% precede each author name, affiliation/snail-mail address and
% e-mail address. Additionally, tag each line of
% affiliation/address with \affaddr, and tag the
% e-mail address with \email.
%
% 1st. author
\alignauthor
Ben Trovato\titlenote{Dr.~Trovato insisted his name be first.}\\
       \affaddr{Institute for Clarity in Documentation}\\
       \affaddr{1932 Wallamaloo Lane}\\
       \affaddr{Wallamaloo, New Zealand}\\
       \email{trovato@corporation.com}
% 2nd. author
\alignauthor
G.K.M. Tobin\titlenote{The secretary disavows
any knowledge of this author's actions.}\\
       \affaddr{Institute for Clarity in Documentation}\\
       \affaddr{P.O. Box 1212}\\
       \affaddr{Dublin, Ohio 43017-6221}\\
       \email{webmaster@marysville-ohio.com}
% 3rd. author
\alignauthor Lars Th{\o}rv{\"a}ld\titlenote{This author is the
one who did all the really hard work.}\\
       \affaddr{The Th{\o}rv{\"a}ld Group}\\
       \affaddr{1 Th{\o}rv{\"a}ld Circle}\\
       \affaddr{Hekla, Iceland}\\
       \email{larst@affiliation.org}
\and  % use '\and' if you need 'another row' of author names
}
% There's nothing stopping you putting the seventh, eighth, etc.
% author on the opening page (as the 'third row') but we ask,
% for aesthetic reasons that you place these 'additional authors'
% in the \additional authors block, viz.
\additionalauthors{Additional authors: John Smith (The Th{\o}rv{\"a}ld Group,
email: {\texttt{jsmith@affiliation.org}}) and Julius P.~Kumquat
(The Kumquat Consortium, email: {\texttt{jpkumquat@consortium.net}}).}
\date{30 July 1999}
% Just remember to make sure that the TOTAL number of authors
% is the number that will appear on the first page PLUS the
% number that will appear in the \additionalauthors section.

\maketitle
\begin{abstract}
This paper ...

\end{abstract}

% A category with the (minimum) three required fields
\category{H.4}{Information Systems Applications}{Miscellaneous}
%A category including the fourth, optional field follows...
\category{D.2.8}{Software Engineering}{Metrics}[complexity measures, performance measures]

\terms{}

\keywords{} % NOT required for Proceedings

\section{Introduction}

Sponsored Search is one of the most applied online advertising types, which provides the major revenue source for adverting services. The common standard in the marketplace is called Cost-Per-Click (CPC). A publisher charges advertisers only when a user clicks on their ads. The efficient allocation of ad space in a Cost-Per-Click system is based on the expected value from each impression on the ad. In order to obtain more payments from advertisers, publishers have to simultaneously maximize estimating Click-Through Rate (CTR) by dynamically learning users feedback and return a set of ranked ads based on expected revenue. 

Since analyzing users feedback through clicks can boost publishers' revenue while it also increases the number of clicks of advertisers' ads and discover the true value for bidding, especially the exploration process. Effectively allocating a set of new ads is essentially a form of the multi-armed bandits problem. If there is no much information about them, using standard relevance matrices does not represent the proper ranking policy. Besides, timeliness is another issue. We need to real-time and dynamically adjust rank decision. Therefore, it indicates that each click has valuable contribution to publishers and advertisers. 

In this paper, first, we conduct experiment to see how publishers increase their revenue by learning CTR from users' clicks and ranking ads. Second, an iterative mechanism is designed to evaluate the value of click over the periods.

\section{Value of a click}
\subsection{Type Changes and  Characters}

Under sponsored search scenario, $N$ denotes number of advertisers and a fixed $M$ shows the number of ad slots where $0<M\leqslant N$. When a user types a query, the search engine matches the keywords of all the advertisers against the query and decides which advertisers are eligible to participate for having their ads displayed to users. The ad $i$ is denoted as vector $\mathbf{x}_{i},i=1,...,N$ containing all information about itself.

For each impression, one of the most common standards for allocating the ad slots to the advertisers depends on the expected revenue for ad $i$. The rank decision or rank action is determined by the multiplication of Click-Through-Rate $p_{i}$ and Cost-Per-Click $b_{i}$. Namely, the advertisers are ordered in their decreasing order of their $p_{i}b_{i}$, and the slots 1 through $M$ are allocated to the top $M$ advertisers in this order, which is a form of the multi-armed bandits problem.

We assume that the click is generated by a mixture of two binomial distributions, where a hidden binary variable is used to to represent the membership. i.e. $S_{i}=0$ denotes it results from the rank bias at rank $i$, and $S_{i}=1$ due to the ad's CTR. This gives the conditional probability of the click on ad $i$:

\begin{equation}
\begin{split}
p(C_{i}|\mathbf{x}_{i}) & =p(C_{i}|\mathbf{x}_{i},S_{i}=1)p(S_{i}=1) \\
 & +p(C_{i}|\mathbf{x}_{i},S_{i}=0)p(S_{i}=0) \\
 & =p(C_{i}|\mathbf{x}_{i})p(S_{i}=1)+p(C_{i}|S_{i}=0)p(S_{i}=0) \\
 & =p_{i}^{C_{i}}(1-p_{i})^{1-C_{i}}\pi_{i}+s_{i}^{C_{i}}(1-s_{i})^{1-C_{i}}(1-\pi_{i})
\end{split}
\end{equation}

where we defined the above three parameters as 
\begin{equation}
p_{i}\equiv p(C_{i}=1|\mathbf{x}_{i}), s_{i}\equiv p(C_{i}=1|S_{i}=0), \pi_{i}\equiv p(S_{i}=1)
\end{equation}

The following algorithm is designed to iteratively update parameters and the posterior probability of clicks, where we attempt to optimize over the values for $\hat{p}_{i}$ given each impression. During the process, the value of click is finally determined by the gain revenue divided by total clicks. In terms of estimating CTR step, we can incorporate more effective and efficient online learning algorithms for future experiments.

Considering the following process: 

\begin{enumerate}
    \item The number of ad slots is given as $M$. Set CPC $b_{i}=1$ for all $N$ advertisers and collect a sample of $N$ (where $0<M\leqslant N$) independent observations $p_{i},i=1,...,N$ following a Beta prior $p_{i}\sim Beta(\alpha,\beta)$, where the value of $\alpha$ and $\beta$ is known. Also, set type parameter $\pi_{i}$ and rank bias $s_{i}$ to that of your click model, and time step $t=1$. The total impressions are $T$ and run the sponsored advertising.
    \item For all ads, record clicks $C_{i}(t)$ and update the posterior probability of mixture CTR model. The charge for ad $i$ is
    \begin{equation}
    \Lambda _{i}(t)=(\hat{p}_{i}\pi_{i}+s_{i}(1-\pi_{i}))b_{i}
    \end{equation}
    at each time $t$.
    \item Re-rank the ads according to $\Lambda _{i}$ value in decreasing order and display the top $M$ of them to the users.

    \item Repeat steps 2 to 4 for $t=2,...,T$
    
    \item Obtain the value of a click $\zeta$ via the gain revenue divided by number of clicks from $t=1$ to $T$, it is expressed as
    \begin{equation} 
    \zeta = \frac{R_{T}-R_{1}}{NC_{T-1}} 
    \end{equation}
    where
    \begin{equation} 
    R_{T}=\sum_{t=1}^{T}\sum_{i=1}^{M}\Lambda _{i}(t)
    \end{equation}
    and
    \begin{equation}
    NC_{T-1}=\sum_{t=1}^{T-1}\sum_{i=1}^{M}C_{i}(t)
    \end{equation}
    
    \item Repeat steps 1 to 6 for 100 times and take the average value of a click.
\end{enumerate}


% The following two commands are all you need in the
% initial runs of your .tex file to
% produce the bibliography for the citations in your paper.
\bibliographystyle{abbrv}
\bibliography{sigproc}  % sigproc.bib is the name of the Bibliography in this case
% You must have a proper ".bib" file
%  and remember to run:
% latex bibtex latex latex
% to resolve all references
%
% ACM needs 'a single self-contained file'!
%
%APPENDICES are optional
%\balancecolumns
\appendix
%Appendix A
\section{Headings in Appendices}
\subsection{Introduction}
\subsection{The Body of the Paper}
\subsubsection{Type Changes and  Special Characters}
\subsubsection{Math Equations}
\paragraph{Inline (In-text) Equations}
\paragraph{Display Equations}
\subsubsection{Citations}
\subsubsection{Tables}
\subsubsection{Figures}
\subsubsection{Theorem-like Constructs}
\subsubsection*{A Caveat for the \TeX\ Expert}
\subsection{Conclusions}
\subsection{Acknowledgments}
\subsection{Additional Authors}
This section is inserted by \LaTeX; you do not insert it.
You just add the names and information in the
\texttt{{\char'134}additionalauthors} command at the start
of the document.
\subsection{References}

% This next section command marks the start of
% Appendix B, and does not continue the present hierarchy
\section{More Help for the Hardy}

\balancecolumns
% That's all folks!
\end{document}
